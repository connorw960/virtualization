\documentclass[11pt]{article}
\usepackage[letterpaper, margin=2cm]{geometry}
\usepackage{titlesec}
\usepackage{mdframed}
\usepackage[dvipsnames]{xcolor} % for color names, must be loaded before tikz
\usepackage{ifthen}
\usepackage{comment}
\usepackage{fancyhdr}
\usepackage{fancyvrb}
\usepackage{titling}
\usepackage{hyperref}
\usepackage{enumitem}
\usepackage{tikz}
\usepackage{amsmath, amssymb, amsthm}

\providecommand{\due}{}
\lhead{Virtualization} \rhead{The University of Texas}
\lfoot{\due} \cfoot{} \rfoot{Page \thepage}
\renewcommand{\footrulewidth}{0.4pt}
\pagestyle{fancy}

% Eliminates the spacing in the title that remains from the empty author section.
\preauthor{}
\postauthor{}

\titleformat{\section}[runin]{\large\bfseries}{\thesection .}{3pt}{}
\titleformat{\subsection}[runin]{\bfseries}{\thesubsection)}{3pt}{}
\renewcommand\thesubsection{\alph{subsection}}

% Defines the solution environment. Toggle solutions between true and false to either show or hide solutions. Also, the solution environment takes an optional argument of arbitrary text to be inserted in the solution header.
\newboolean{solutions}
\setboolean{solutions}{true}
\ifthenelse{\boolean{solutions}}
{\newenvironment{solution}{\begin{mdframed}[skipbelow=0pt, linecolor=White, backgroundcolor=Green!10]\textbf{Solution:}}{\end{mdframed}}}
{\excludecomment{solution}}

\allowdisplaybreaks

\begin{document}

\title{Virtualization\\Pre-lab Questions --- \#4}
\date{\due}

\maketitle

\noindent \textbf{Point total:} 4
\\ All homework assignments are weighted equally in the final grade. Point values are unique to each lab assignment.

\textbf{Note:} For all problems which ask you to explain your reasoning or show your work, you do not need to show every step of each calculation, but the answer should include an explanation \emph{written with words} of what you did.  Even when work is not required to be shown, it’s a good idea to include anyways so that you can earn partial credit.

\section{Question [2 points]}

An easy way we can help prevent code duplication is by using macros in functions. We saw one example of this in Lab 1 with the \texttt{VMM\_GUEST} macro. In many of the hypercalls we have in JOS, we need to differentiate when we are running in the guest or host. Give an example of a C program that adds up all the numbers in an array when the macro is set, and multiplies them when the macro is not set.

\begin{solution}
int main()
{
    int array[] = {1, 2, 3, 4, 5, 6, 7, 8, 9, 10};

    / Get the total size of the array in bytes
    int sizeInBytes = sizeof(array);

    // Get the size of one element in the array
    int elementSize = sizeof(array[0]);

    // Calculate the number of elements in the array
    int length = sizeInBytes / elementSize;

    modifyArray(array, output, length);

    for(size_t i =0; i < length; i++)
    {
        printf("Output: \%f",output);
    }
}

void modifyArray(int array, double *output, int length)
{
#ifndef ADD_ARRAY 
    for(int i=0;i<size;i++)
    {
        output *= array[i];
    }
#else
    output -= 1;
    for(int i=0;i<size;i++)
    {
        output += array[i];
    }
#endif
}
\end{solution}


\section{Question [1 point]}

In this lab, you will write some code to implement demand paging for JOS VMs. What is a buffer cache? What is demand paging? How do these methods impact the performance of the file system?

\begin{solution}
A buffer cache is memory that is used to store data that needs to be access frequently.
This reduces the number of times the system has to read from the disk with an I/O instruction since this is very slow.
This reduces the latency that is caused by slow I/O but requires additional memory.
Demand paging is used to load memory pages into RAM on "demand" or as needed. Instead of preloading the full data to memory.
This reduces the memory required to run large programs, it can also speed up the startup time of large programs.
This however increases page faults since the entire program is not loaded at start up. If the program is on a disk, the increased page faults will increase the need to access the disk and increase latency. 
If Demand paging is used with a buffer cache it can reduce the number of times the system needs to access the disk and improve performance.
\end{solution}


\section{Question [1 point]}

After completing the lab, provide a simple diagram (you can just use function names and arrows) 
of the workflow of the functions that get called when a guest VM tries to read from a file.

\begin{solution}
hostfs_ipc()->
ipc_host_send()->
ipc_host_recv();
\end{solution}


\end{document}
