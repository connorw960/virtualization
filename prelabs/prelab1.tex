\documentclass[11pt]{article}
\usepackage[letterpaper, margin=2cm]{geometry}
\usepackage{titlesec}
\usepackage{mdframed}
\usepackage[dvipsnames]{xcolor} % for color names, must be loaded before tikz
\usepackage{ifthen}
\usepackage{comment}
\usepackage{fancyhdr}
\usepackage{fancyvrb}
\usepackage{titling}
\usepackage{hyperref}
\usepackage{enumitem}
\usepackage{tikz}
\usepackage{amsmath, amssymb, amsthm}

\providecommand{\due}{}
\lhead{Virtualization} \rhead{The University of Texas}
\lfoot{\due} \cfoot{} \rfoot{Page \thepage}
\renewcommand{\footrulewidth}{0.4pt}
\pagestyle{fancy}

% Eliminates the spacing in the title that remains from the empty author section.
\preauthor{}
\postauthor{}

\titleformat{\section}[runin]{\large\bfseries}{\thesection .}{3pt}{}
\titleformat{\subsection}[runin]{\bfseries}{\thesubsection)}{3pt}{}
\renewcommand\thesubsection{\alph{subsection}}

% Defines the solution environment. Toggle solutions between true and false to either show or hide solutions. Also, the solution environment takes an optional argument of arbitrary text to be inserted in the solution header.
\newboolean{solutions}
\setboolean{solutions}{true}
\ifthenelse{\boolean{solutions}}
{\newenvironment{solution}{\begin{mdframed}[skipbelow=0pt, linecolor=White, backgroundcolor=Green!10]\textbf{Solution:}}{\end{mdframed}}}
{\excludecomment{solution}}

\allowdisplaybreaks

\begin{document}

\title{Virtualization\\Pre-lab Questions --- \#1}
\date{\due}

\maketitle

\noindent \textbf{Point total:} 7
\\ All homework assignments are weighted equally in the final grade. Point values are unique to each lab assignment.

\textbf{Note:} For all problems which ask you to explain your reasoning or show your work, you do not need to show every step of each calculation, but the answer should include an explanation \emph{written with words} of what you did.  Even when work is not required to be shown, it’s a good idea to include anyways so that you can earn partial credit.

\section{Question [1 point]}

What does the \texttt{cpuid} assembly instruction do in this function?

\begin{solution}
The overall purpose of this instruction is to gather information about the CPU based off of the `info' flag 
(what it is set to determines what info is retrieved). It will populate various registers with the 
desired info. Then it takes the results from the 
registers specified and assigns them to the variables eax, ebx, ecx, and edx. 
\end{solution}


\section{Question [1 point]}

How is the function providing values back to you to use?

\begin{solution}
The function is assigning the values into pointers we fed into the function. We will be able to 
access these pointers later on since they were passed by reference. 
\end{solution}


\section{Question [1 point]}

What are the results of \texttt{eax}, \texttt{ecx}, and \texttt{ebx} values in hexadecimal?

\begin{solution}
    \texttt{\newline
    (gdb) p/x eax
    0x4 \newline
    (gdb) p/x ebx
     0x756e6547 \newline
    (gdb) p/x ecx
    0x6c65746e\newline
    (gdb) p/x edx
    0x49656e69\newline
    }
\end{solution}


\section{Question [1 point]}

Now examine the values of these variables as strings. 
Hint: look at the values in hexadecimal and translate them to strings, 
in the order \texttt{ebx}, \texttt{edx}, \texttt{ecx}. What do you observe? 
The \href{https://en.wikipedia.org/wiki/CPUID}{Wikipedia page} for the cpuid instruction 
may help you interpret this output.

\begin{solution}
The string evaluates to GenuineIntel.
\end{solution}


\section{Question [1 point]}

There is a reference in each \texttt{Env} struct for another struct called 
\texttt{VmxGuestInfo}. What kind of information does this struct hold?

\begin{solution}
It contains the physical memory size, the vmcs pointer referenced in later questions, an exception bitmap, an 
I/O bitmap, and an MSR load/store area. Essentially, it contains all of the info related to a guest VM  
and what might be needed to manage it. 
\end{solution}


\section{Question [1 point]}

From \href{https://www.cs.utexas.edu/~vijay/cs378-f17/projects/64-ia-32-architectures-software-developer-vol-3c-part-3-manual.pdf}{this Intel guide}, 
find out what the vmcs pointer in this struct stands for, and what it purpose it serves.

\begin{solution}
VMCS stands for `Virtual Machine Control Structure'. It is meant to be used by the hypervisor to manage the 
state of the VM. A VMM can use a separate VMCS for each VM, or each virtual processor. It essentially has 
has all of the info that allow the VMM to control the VM and is organized into 6 logical groups. 
Guest-state area, host-state area, VM-execution control fields, VM-exit control fields, VM-entrry control 
fields, and VM-exit information fields. 
\end{solution}


\section{Question [1 point]}

What assembly instruction initializes the \texttt{vmcs} pointer? 
In other words, how do we change the \texttt{vmcs} pointer?

\begin{solution}
First, memory needs to be allocated for the new VMCS. Then, VMCLEAR is used to initialize the VMCS 
and set all of its fields to default values. VMPTRLD is then used to load the new pointer. After that, 
the commands VMLAUNCH, VMREAD, VMRESUME, and VMWRITE are used to modify the current VMCS. 
\end{solution}


\end{document}

