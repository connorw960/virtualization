\documentclass[11pt]{article}
\usepackage[letterpaper, margin=2cm]{geometry}
\usepackage{titlesec}
\usepackage{mdframed}
\usepackage[dvipsnames]{xcolor} % for color names, must be loaded before tikz
\usepackage{ifthen}
\usepackage{comment}
\usepackage{fancyhdr}
\usepackage{fancyvrb}
\usepackage{titling}
\usepackage{hyperref}
\usepackage{enumitem}
\usepackage{tikz}
\usepackage{amsmath, amssymb, amsthm}

\providecommand{\due}{}
\lhead{Virtualization} \rhead{The University of Texas}
\lfoot{\due} \cfoot{} \rfoot{Page \thepage}
\renewcommand{\footrulewidth}{0.4pt}
\pagestyle{fancy}

% Eliminates the spacing in the title that remains from the empty author section.
\preauthor{}
\postauthor{}

\titleformat{\section}[runin]{\large\bfseries}{\thesection .}{3pt}{}
\titleformat{\subsection}[runin]{\bfseries}{\thesubsection)}{3pt}{}
\renewcommand\thesubsection{\alph{subsection}}

% Defines the solution environment. Toggle solutions between true and false to either show or hide solutions. Also, the solution environment takes an optional argument of arbitrary text to be inserted in the solution header.
\newboolean{solutions}
\setboolean{solutions}{true}
\ifthenelse{\boolean{solutions}}
{\newenvironment{solution}{\begin{mdframed}[skipbelow=0pt, linecolor=White, backgroundcolor=Green!10]\textbf{Solution:}}{\end{mdframed}}}
{\excludecomment{solution}}

\allowdisplaybreaks

\begin{document}

\title{Virtualization\\Pre-lab Questions --- \#1}
\date{\due}

\maketitle

\noindent \textbf{Point total:} 7
\\ All homework assignments are weighted equally in the final grade. Point values are unique to each lab assignment.

\textbf{Note:} For all problems which ask you to explain your reasoning or show your work, you do not need to show every step of each calculation, but the answer should include an explanation \emph{written with words} of what you did.  Even when work is not required to be shown, it’s a good idea to include anyways so that you can earn partial credit.

\section{Question [1 point]}

What does the \texttt{cpuid} assembly instruction do in this function?

\begin{solution}
Your answer here.
\end{solution}


\section{Question [1 point]}

How is the function providing values back to you to use?

\begin{solution}
Your answer here.
\end{solution}


\section{Question [1 point]}

What are the results of \texttt{eax}, \texttt{ecx}, and \texttt{ebx} values in hexadecimal?

\begin{solution}
Your answer here.
\end{solution}


\section{Question [1 point]}

Now examine the values of these variables as strings. Hint: look at the values in hexadecimal and translate them to strings, in the order \texttt{ebx}, \texttt{edx}, \texttt{ecx}. What do you observe? The \href{https://en.wikipedia.org/wiki/CPUID}{Wikipedia page} for the cpuid instruction may help you interpret this output.

\begin{solution}
Your answer here.
\end{solution}


\section{Question [1 point]}

There is a reference in each \texttt{Env} struct for another struct called \texttt{VmxGuestInfo}. What kind of information does this struct hold?

\begin{solution}
Your answer here.
\end{solution}


\section{Question [1 point]}

From \href{https://www.cs.utexas.edu/~vijay/cs378-f17/projects/64-ia-32-architectures-software-developer-vol-3c-part-3-manual.pdf}{this Intel guide}, find out what the vmcs pointer in this struct stands for, and what it purpose it serves.

\begin{solution}
Your answer here.
\end{solution}


\section{Question [1 point]}

What assembly instruction initializes the \texttt{vmcs} pointer? In other words, how do we change the \texttt{vmcs} pointer?

\begin{solution}
Your answer here.
\end{solution}


\end{document}

